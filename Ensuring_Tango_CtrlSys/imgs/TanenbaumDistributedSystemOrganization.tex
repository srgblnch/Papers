% PSTricks TeX macro
% Title: /home/serguei/src/Papers/Ensuring_Tango_CtrlSys/imgs/TanenbaumDistributedSystemOrganization.dia
% Creator: Dia v0.97.2
% CreationDate: Sun Aug 11 19:16:03 2013
% For: serguei
% \usepackage{pstricks}
% The following commands are not supported in PSTricks at present
% We define them conditionally, so when they are implemented,
% this pstricks file will use them.
\ifx\setlinejoinmode\undefined
  \newcommand{\setlinejoinmode}[1]{}
\fi
\ifx\setlinecaps\undefined
  \newcommand{\setlinecaps}[1]{}
\fi
% This way define your own fonts mapping (for example with ifthen)
\ifx\setfont\undefined
  \newcommand{\setfont}[2]{}
\fi
\pspicture(3.900000,-19.100000)(24.100000,-5.386312)
\psscalebox{1.000000 -1.000000}{
\newrgbcolor{dialinecolor}{0.000000 0.000000 0.000000}%
\psset{linecolor=dialinecolor}
\newrgbcolor{diafillcolor}{1.000000 1.000000 1.000000}%
\psset{fillcolor=diafillcolor}
\psset{linewidth=0.100000cm}
\psset{linestyle=solid}
\psset{linestyle=solid}
\setlinejoinmode{0}
\newrgbcolor{dialinecolor}{0.000000 0.000000 0.000000}%
\psset{linecolor=dialinecolor}
\pspolygon(4.000000,7.000000)(4.000000,17.000000)(9.690000,17.000000)(9.690000,7.000000)
\psset{linewidth=0.100000cm}
\psset{linestyle=solid}
\psset{linestyle=solid}
\setlinejoinmode{0}
\newrgbcolor{dialinecolor}{0.000000 0.000000 0.000000}%
\psset{linecolor=dialinecolor}
\pspolygon(11.000000,7.000000)(11.000000,17.000000)(16.690000,17.000000)(16.690000,7.000000)
\psset{linewidth=0.100000cm}
\psset{linestyle=solid}
\psset{linestyle=solid}
\setlinejoinmode{0}
\newrgbcolor{dialinecolor}{0.000000 0.000000 0.000000}%
\psset{linecolor=dialinecolor}
\pspolygon(18.000000,7.000000)(18.000000,17.000000)(23.690000,17.000000)(23.690000,7.000000)
\psset{linewidth=0.100000cm}
\psset{linestyle=solid}
\psset{linestyle=solid}
\setlinejoinmode{0}
\newrgbcolor{dialinecolor}{1.000000 1.000000 1.000000}%
\psset{linecolor=dialinecolor}
\pspolygon*(5.000000,11.000000)(5.000000,13.000000)(23.000000,13.000000)(23.000000,11.000000)
\newrgbcolor{dialinecolor}{0.000000 0.000000 0.000000}%
\psset{linecolor=dialinecolor}
\pspolygon(5.000000,11.000000)(5.000000,13.000000)(23.000000,13.000000)(23.000000,11.000000)
\setfont{Helvetica}{0.800000}
\newrgbcolor{dialinecolor}{0.000000 0.000000 0.000000}%
\psset{linecolor=dialinecolor}
\rput(14.000000,12.000000){\psscalebox{1 -1}{Tango Middleware}}
\psset{linewidth=0.100000cm}
\psset{linestyle=solid}
\psset{linestyle=solid}
\setlinejoinmode{0}
\newrgbcolor{dialinecolor}{1.000000 1.000000 1.000000}%
\psset{linecolor=dialinecolor}
\pspolygon*(5.000000,8.000000)(5.000000,10.000000)(23.000000,10.000000)(23.000000,8.000000)
\newrgbcolor{dialinecolor}{0.000000 0.000000 0.000000}%
\psset{linecolor=dialinecolor}
\pspolygon(5.000000,8.000000)(5.000000,10.000000)(23.000000,10.000000)(23.000000,8.000000)
\setfont{Helvetica}{0.800000}
\newrgbcolor{dialinecolor}{0.000000 0.000000 0.000000}%
\psset{linecolor=dialinecolor}
\rput(14.000000,9.000000){\psscalebox{1 -1}{Distributed application}}
\psset{linewidth=0.100000cm}
\psset{linestyle=solid}
\psset{linestyle=solid}
\setlinejoinmode{0}
\newrgbcolor{dialinecolor}{0.000000 0.000000 0.000000}%
\psset{linecolor=dialinecolor}
\pspolygon(5.000000,14.000000)(5.000000,16.000000)(9.000000,16.000000)(9.000000,14.000000)
\setfont{Helvetica}{0.800000}
\newrgbcolor{dialinecolor}{0.000000 0.000000 0.000000}%
\psset{linecolor=dialinecolor}
\rput(7.000000,15.000000){\psscalebox{1 -1}{Local OS}}
\psset{linewidth=0.100000cm}
\psset{linestyle=solid}
\psset{linestyle=solid}
\setlinejoinmode{0}
\newrgbcolor{dialinecolor}{0.000000 0.000000 0.000000}%
\psset{linecolor=dialinecolor}
\pspolygon(12.000000,14.000000)(12.000000,16.000000)(16.000000,16.000000)(16.000000,14.000000)
\setfont{Helvetica}{0.800000}
\newrgbcolor{dialinecolor}{0.000000 0.000000 0.000000}%
\psset{linecolor=dialinecolor}
\rput(14.000000,15.000000){\psscalebox{1 -1}{Local OS}}
\psset{linewidth=0.100000cm}
\psset{linestyle=solid}
\psset{linestyle=solid}
\setlinejoinmode{0}
\newrgbcolor{dialinecolor}{0.000000 0.000000 0.000000}%
\psset{linecolor=dialinecolor}
\pspolygon(19.000000,14.000000)(19.000000,16.000000)(23.000000,16.000000)(23.000000,14.000000)
\setfont{Helvetica}{0.800000}
\newrgbcolor{dialinecolor}{0.000000 0.000000 0.000000}%
\psset{linecolor=dialinecolor}
\rput(21.000000,15.000000){\psscalebox{1 -1}{Local OS}}
\psset{linewidth=0.200000cm}
\psset{linestyle=solid}
\psset{linestyle=solid}
\setlinecaps{0}
\newrgbcolor{dialinecolor}{0.000000 0.000000 0.000000}%
\psset{linecolor=dialinecolor}
\psline(4.000000,19.000000)(24.000000,19.000000)
\psset{linewidth=0.200000cm}
\psset{linestyle=solid}
\psset{linestyle=solid}
\setlinecaps{0}
\newrgbcolor{dialinecolor}{0.000000 0.000000 0.000000}%
\psset{linecolor=dialinecolor}
\psline(7.000000,17.000000)(7.000000,19.000000)
\psset{linewidth=0.200000cm}
\psset{linestyle=solid}
\psset{linestyle=solid}
\setlinecaps{0}
\newrgbcolor{dialinecolor}{0.000000 0.000000 0.000000}%
\psset{linecolor=dialinecolor}
\psline(14.000000,17.000000)(14.000000,19.000000)
\psset{linewidth=0.200000cm}
\psset{linestyle=solid}
\psset{linestyle=solid}
\setlinecaps{0}
\newrgbcolor{dialinecolor}{0.000000 0.000000 0.000000}%
\psset{linecolor=dialinecolor}
\psline(21.000000,17.000000)(21.000000,19.000000)
\setfont{Helvetica}{0.800000}
\newrgbcolor{dialinecolor}{0.000000 0.000000 0.000000}%
\psset{linecolor=dialinecolor}
\rput[l](5.000000,4.000000){\psscalebox{1 -1}{}}
\setfont{Helvetica}{0.800000}
\newrgbcolor{dialinecolor}{0.000000 0.000000 0.000000}%
\psset{linecolor=dialinecolor}
\rput(7.000000,6.000000){\psscalebox{1 -1}{Machine A}}
\setfont{Helvetica}{0.800000}
\newrgbcolor{dialinecolor}{0.000000 0.000000 0.000000}%
\psset{linecolor=dialinecolor}
\rput(7.000000,6.800000){\psscalebox{1 -1}{}}
\setfont{Helvetica}{0.800000}
\newrgbcolor{dialinecolor}{0.000000 0.000000 0.000000}%
\psset{linecolor=dialinecolor}
\rput(14.000000,6.000000){\psscalebox{1 -1}{Machine B}}
\setfont{Helvetica}{0.800000}
\newrgbcolor{dialinecolor}{0.000000 0.000000 0.000000}%
\psset{linecolor=dialinecolor}
\rput(21.000000,6.000000){\psscalebox{1 -1}{Machine C}
\endpspicture