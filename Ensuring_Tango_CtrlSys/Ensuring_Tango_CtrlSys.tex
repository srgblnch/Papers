%\documentclass[a4paper,10pt]{article}
\documentclass[10pt,a4paper,twoside]{llncs}
%\usepackage[utf8x]{inputenc}
\usepackage[T1]{fontenc}
\usepackage[latin1]{inputenc}
\usepackage{amsmath,amsfonts,amssymb}%\usepackage{amsmath,amsfonts,amsthm,amssymb}
\usepackage{graphicx}
\usepackage{algorithmic,algorithm}
\usepackage{tikz}
\usetikzlibrary{matrix}

\pagestyle{headings}%page numbers

%%%%%%%%%%%%%%%%%%%%%%%%%%% VERSIONES %%%%%%%%%%%%%%%%%%%%%%%%%%%%%%%%%%%
\usepackage{gitinfo}
\newcommand{\version}{github.Papers: \gitCommitterDate\;(revision \gitAbbrevHash) }
\newcommand{\todo}[1]{\texttt{\color{red}TODO:} ``\emph{#1}''}
\newcommand{\fixme}[1]{\texttt{\color{red}FIXME:} ``\emph{#1}''}
%%%%%%%%%%%%%%%%%%%%%%%%%%%%%%%%%%%%%%%%%%%%%%%%%%%%%%%%%%%%%%%%%%%%%%%%%
\newcommand{\tango}{\textsc{Tango} }

%opening
\title{Ensuring \tango Control System}
\author{Sergi Blanch-Torn\'e\inst{1}, Ramiro Moreno Chiral\inst{2}}
 \institute{
 Escola Polit\`ecnica Superior, Universitat de Lleida. Spain.\\
 \email{\tt sblanch@alumnes.udl.es}
 \and 
 Departament de Matem\`atica. Universitat de Lleida. Spain.\\
 \email{\tt ramiro@matematica.udl.es}
 }

%%% Definiciones matem\'aticas especiales

\newcommand{\Z}{\ensuremath{\mathbb{Z}}}%                       Enteros
\newcommand{\Q}{\ensuremath{\mathbb{Q}}}%                       Racionales
\newcommand{\A}{\ensuremath{\mathcal{A}_{2}}}%                   Plano Af\'{\i}n
\newcommand{\Proy}{\ensuremath{\mathcal{P}_{2}}}%                Plano Proyectivo
\newcommand{\Jacob}{\ensuremath{\mathcal{J}_{2}}}%               Plano Jacobiano
\newcommand{\K}{\ensuremath{\mathbb{K}}}%                       Cuerpo en general
\newcommand{\F}{\ensuremath{\mathbb{F}}}%                       Cuerpo finito en general
\newcommand{\Fp}{\ensuremath{\mathbb{F}_p}}%                    Cuerpo finito de orden p (primo)
\newcommand{\EFp}{\ensuremath{E(\mathbb{F}_p)}}%                Curva el\'\{i}ptica sobre un cuerpo finito de orden p (primo)
\newcommand{\EFq}{\ensuremath{E(\mathbb{F}_q)}}%                Curva el\'\{i}ptica sobre un cuerpo finito
\newcommand{\Fm}{\ensuremath{\mathbb{F}_{2^m}}}%                Cuerpo finito de caractar\'{\i}stica 2, grado m
\newcommand{\EFm}{\ensuremath{E(\mathbb{F}_p)}}%                Curva el\'\{i}ptica sobre un cuerpo finito de caractar\'{\i}stica 2, grado m
\newcommand{\Fq}{\ensuremath{\mathbb{F}_q}}%                    Idem id. q (q=p^m, p primo y m entero pos.)
\newcommand{\Zn}[1]{\ensuremath{\mathbb{Z}/#1\mathbb{Z}}}%      Anillo de los enteros mod n
\newcommand{\PaI}{\ensuremath{\mathcal{O}}}%                    Punto en el Infinito
\newcommand{\PaIe}{\ensuremath{\mathcal{O}_{E}}}%               Punto en el Infinito de la curva

%%% Algorithm customization:
%\floatname{algorithm}{Procedure}%Rename the text ``Algorithm''
\renewcommand{\algorithmicrequire}{\textbf{Input:}}
\renewcommand{\algorithmicensure}{\textbf{Output:}}
%%% end algorithm

\begin{document}

\maketitle
\begin{center}
 \today\\
 \version
\end{center}


\begin{abstract}\footnote{Partially supported by grants MTM2010-21580-C02-01 (Spanish Ministerio de Ciencia e Innovaci\'on), 2009SGR-442 (Generalitat de Catalunya).}
\todo{embedded cryptography}

\todo{Ensuring \tango must be like http\emph{s}. Transparent as possible from the current usage.}
\\\\    
{\bf Keywords:} Cryptography, Elliptic Curves, Distributed Systems, SCADA, Controls system, Synchrotron
\end{abstract}

%%%%%
%
\section{Introduction \label{sec:intro}}

\todo{What is \tango?}

\todo{Distributed systems transparencies \cite{TanenbaumDistr} that \tango complains, and which are not}

\todo{Go further that the Locking/Access control}

\todo{Why to secure it? Trust in a peripheral firewalls is not enough.}

\todo{Embedded in instrumentation, limited calculation capacity (it must behave indistinguishable if it's a huge server or an embedded board), limited bandwidth (Don't increase the current needs significantly): very good candidate for elliptic curves, generalized Rijndael and stream cipher.}

\todo{The price of the information and the balance between the cost to ensure and the value of the ensured goods. Security levels: Open, confidential, Secret, Top Secret. (remember the German standard on this levelling).}

%%%%%
%
\section{Identifying scenarios \label{sec:scenarios}}

\todo{In terms of security threads, which is more representative from \cite{SecEngRossAnderson} for the current use case? Hospital, Bank, Military Base. Practical paranoia \cite{PractCryptoSchneier}}

\todo{Key distribution protocols \cite{SecEngRossAnderson} sec.3.7.2}

%%%%%
%
\subsection{Access Control}

\todo{Agent authentication in a distributed system}

\todo{Ensuring communication between agents and between those agents with the user interfaces. \emph{Command}, \emph{read} and \emph{write} operations; \emph{Properties} modifications and changes application. This can be compared with \emph{RFID} communication between card and readers, but adding communication in between the agents}

\todo{\tango database access control}

\todo{Ensuring between instrumentation and the agents out of the scope of this paper}

\todo{Trusted Computing and Hardware protections}
%%%%%
%
\subsection{Secret sharing}

\todo{multicast, events and the other features that must be secured}

%%%%%
%
\subsection{Brainstorming attacks}

%%%%%
%
\subsubsection{Passive attacks}

\todo{Eavesdropping (Passive attacks) and Men-in-the-middle (active attacks) between agents.}

\todo{Noise to block an alarm transmission}

%%%%%
%
\subsubsection{Active attacks}

\todo{Break the public face, web site or gui}

\todo{Supplant agents.}

%%%%%
%
\subsection{Intrusion Detection}

%%%%%
%
\section{Zero-knowledge proof for authentication \label{sec:auth}}

\todo{The agents in the distributed system must be authenticated to be sure that they hasn't been supplanted}

%%%%%
%
\section{Protocols}

\todo{protocol layers \cite{Schneier:1995:ACP:572932}}

%%%%%
%
\subsection{Communication hybrid schema \label{sec:intercom}}

\todo{Pubkey to agreed a season key as the usual hybrid systems}

\todo{Use the Symmetric key to seed a shared PseudoRandomGenerator as a key for a stream cipher of transmitted data and listened data between talkers}

\todo{\emph{PseudoRandomGenerator} (PRG), can be use the KeyDerivation of the Rijndael or better other possible alternatives}

%%%%%
%
\section{Conclusion \label{sec:conclusions}}

\bibliographystyle{ieeetr}
\bibliography{../bibtex/sblanch.bib,../bibtex/standards.bib,../bibtex/ecc.bib,../bibtex/isogeny.bib,../bibtex/books.bib,../bibtex/crypto.bib,../bibtex/rfc.bib}

\end{document}
